\documentclass{article}

\begin{document}
\title{Dangerous Doughnuts (Team D) - Team Organisation Plan} %Is it possible to put ``Team organisation plan'' at a new line???
\author{Horatiu-Sorin Bota \\
		Konstantinos Dadamis \\
		Paweł Drewniak \\
		Motiejus Jakštys \\
		Andrew MacLeod} %Be careful with the unicode stuff

\date{28 September 2011}
\maketitle

\section{Introduction}
\label{intro}

This is the Dangerous Doughnuts' organisation plan for the 1st Team
Project of the PSD3 course. It consists of \ref{lastSection}
chapters. Chapter \ref{intro} is an introduction to this document,
chapter \ref{role} includes the roles of each member of the team, the decision making process is kept in chapter \ref{auth} and chapter \ref{com} deals with the ways the members will communicate. The last two chapters of this document contain the information management process and the underlying risks in the team's organisational decisions respectively.

DO WE NEED DOCUMENT STATUS AND SCHEDULE????

%\subsection{Document Status and Schedule}
%Describe the status, including goals and dates, for production and
%revision of the document.  Documentation is often generated
%incrementally and iteratively. If this is the case for this document,
%also summarise here the planned updates and their release dates.

\section{Roles}
\label{role}

Initially, the team members will have loosely defined roles (MUDs model). The first plan is that the toolsmiths will be both Paweł Drewniak and Motiejus Jakštys, providing the team with the necessary tools for the development process, such as the version control system. Horatiu-Sorin Bota and Andrew MacLeod will be responsible of the quality assurance, gathering of the requirements and providing documentation. Finally, the coding will be assigned to Paweł Drewniak, Motiejus Jakštys and Konstantinos Dadamis. The roles given will not be strict and whichever section turns out to be more demanding, more team members will be assigned to it.  

This first plan is only a draft and as time progresses, there will be a better understanding of each member's abilities and the team organisation will evolve with new roles emerging. 

%%%%%%%%%%%%%%%%%%%%%%%%%%%%%%%%%%%%%%%%%%%%%%%%%%%%%%%%%%%%%%%%%%%%%%%%%%%%%%

\section{Authority}
\label{auth}

As the MUDs team organisation model was selected, there is not a project leader inside the team. Instead, a collective management approach was preferred. Each member of the team has his own ideas and propositions and everything is discussed during the formal or informal meetings which take place on a daily basis. A democratic voting system is employed when any disputes arise. This organisation model requires constant and instant communication between the team members, so an appropriate communication plan has been followed.

%%%%%%%%%%%%%%%%%%%%%%%%%%%%%%%%%%%%%%%%%%%%%%%%%%%%%%%%%%%%%%%%%%%%%%%%%%%%%%

\section{Communication}
\label{com}

As mentioned above, the team will arrange meetings on a daily basis. All of the team members attend the same courses at the Computing Science department of the University of Glasgow(too much????????), so the meetings can be scheduled after the end of the lectures. They will take place either at the Computing Science lab, or at a cafe(punctuation plzzzzz) near the University.

Another means of communication will be Google Wave. Google Wave is a web-based computing platform and communications protocol which is centered on online real-time collaborative editing. Its features benefit the Dangerous Doughnuts communications as it is specifically designed for this kind of project(or projects????), allowing all of the members to chat, brainstorm, attach files and edit everything in the same wave. In addition, all the changes to the waves are recorded, so there is a full history of every message written or edited. Consequently, if a member is offline when a certain message is sent, he will still be able to retrieve the message whenever he goes online.

Furthermore, communication is also achieved through email, Skype, phone calls and SMS messaging.

%%%%%%%%%%%%%%%%%%%%%%%%%%%%%%%%%%%%%%%%%%%%%%%%%%%%%%%%%%%%%%%%%%%%%%%%%%%%%%

\section{Information Management}
\label{info}

All the information produced by the team is stored in a central repository at github.com. Using a central repository allows the team to work collaboratively in the project, check each member's work and combine everything to make the final version of the project function. In addition, Git is also employed to help the development of the project. Git is a distributed revision control system, helping the team keep track of software revisions. Its distributed feature allows all the members to work on the project without necessarily being connected to the Internet.

WHO CAN USE THE INFORMATION??????? WHY DID WE PREFER GIT?????

%Where is information kept?  How and when will it distributed?  Who can
%use it?

%%%%%%%%%%%%%%%%%%%%%%%%%%%%%%%%%%%%%%%%%%%%%%%%%%%%%%%%%%%%%%%%%%%%%%%%%%%%%%

\section{Organisational Risks}
\label{lastSection}

The decisions the team made in terms of organisation were made after careful consideration and in order to take advantage of the team's full potential. However there are some underlying risks. One serious risk that needs to be taken into account is the possibility that a serious dispute might arise. In this case, the team members won't be able to agree on a certain approach and precious time will be lost while arguing. This could be avoided if an organisation plan with a Chief Programmer existed where he would take the final decision. The Dangerous Doughnuts members are hoping to avoid such kind of disputes by arguing constructively and backing down when they need to.

Another risk that the team had to take was that the member roles are not strictly set and there might be a confusion between the members about which task is assigned to whom. In this case, two members might end up working on the same task or there could even be a task without someone assigned to it. Again, this could be avoided by using the role-based organisation plan. Dangerous Doughnuts are willing to overcome this potential obstacle with(is ``with'' right???) good communication.


%==============================================================================
\bibliographystyle{plain}
\bibliography{example}
\end{document}
